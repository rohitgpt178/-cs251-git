\documentclass{article}

\title{Movie Review}
\author{Nobody}
\date{Today}
\begin{document}
\maketitle

\LARGE{ANAND(1971) \& GOLMAAL(1979)}

\Large{Anand}


Cast\newline
Rajesh Khanna as Anand Sehgal\newline
Amitabh Bachchan as Dr. Bhaskar Banerjee a.k.a. Babu Moshai\newline
Rekha as Meera (Special Appearance)\newline
Sumita Sanyal as Renu\newline
Ramesh Deo as Dr. Prakash Kulkarni a.k.a. Dost\newline
\newline

What an outstanding movie!! I have heard all the prior generation people rave about this movie, so, I decided to check this movie out myself. I only have faint memories of having watched parts of this movie from my mom's lap when she and dad were watching this in the theater. The other reason why I decided to check this out was a Super-bowl half-time debate on whether Amitabh was better than Rajesh Khanna. I could not participate in this debate for two reasons: first, I was eagerly awaiting another "wardrobe malfunction" for one of the cheerleaders and secondly, I only had memories of one Rajesh Khanna movie, Haathi Mere Saathi. I remember having enjoyed it very much as a child. But that alone was not enough to quantify anything. The more recent performances of AB were fresh in my mind, but after having seen this movie, I decided that Rajesh had a class of his own. His chirpy performance in this movie is really unparalleled! What an amazing performance! Amitabh, being more junior, has not equaled Rajesh, but has done his share very well. Thus, even after watching this movie, the debate will continue.

What's New? What can possibly be new in an old movie? Guess what? there is plenty for the younger generation to take away! There is no education in the movies these days, whatsoever, except perhaps bedroom or bar sequences. The inadequacies in the field of medicine are so nicely brought forward by this movie, which is certainly over 30 years old! Now, I can understand why Munnabhai MBBS was such a hit. It had so beautifully adapted from this movie to match the present generation. Kudos to Vidhu Vinod Chopra and Sanjay Dutt (\& Kamal Hassan too) for carrying this forward.

Noticeable: "BaaabuMushaai", the nick name for Amitabh, as heard from Rajesh throughout the movie, will ring in your ears even hours after you have seen the six letters "The End" on the screen. There really is no end to such people! Some sequences were simply amazing. Those that stood apart in my mind were the last scene, with a tape that had a significant pause in between; the moun-vrath guru, who so symbolically said that there is so much more than the decaying body to Anand's soul; then of course the Munirbhai sequences and the eventual backfiring of this strategy and so many more! The songs were so gentle and heart warming! The comic timing of Rajesh Khanna was simply amazing! Verdict Present day filmmakers really need to rework their brains and start thinking much much better! There is much more to Hindi cinema than just skin-show and catchy item number songs. This is a MUST-WATCH movie! I did not think so when others told me, but having experienced it myself, I believe them! I am gonna check out the other MUST-WATCH movies prescribed by the previous generation. 


\Large{Golmaal}

Cast:\newline
    Amol Palekar - Ramprasad Sharma/Lakshmanprasad Sharma\newline
    Utpal Dutt - Bhawani Sankar\newline
    Bindiya Goswami - Urmila\newline
    David - Doctor Kedar Mama\newline
    Deven Verma - as himself and Ramprasad's best friend\newline
\newline
As much as Delhi Belly is not my sort of comedy, Hrishikesh Mukherjee's Golmaal ("confusion") is a movie made for me.  Light, smart, and downright funny, Golmaal is a story of clever people improvising their way through a thoroughly ridiculous series of deceptions.  It is a simply delightful movie, similar in feel to Hrishida's equally charming Chupke chupke. 

With his freshly minted degree in accounting, Ram Prasad Sharma (Amol Palekar) needs a job.  To land a position with the stern businessman Bhavani Shankar (Utpal Dutt), Ram plays up his studious, serious side - dressing in traditional clothing, spouting aphorisms about the value of hard work, and concealing his passion for sports.  After Bhavani spots Ram skiving off work to watch a hockey match, the quick-thinking Ram invents a no-good, layabout twin, whom he calls (of course) Lakshman, and convinces Bhavani that the fellow he saw at the match was Lakshman, not Ram.  Bhavani, wanting to favor his protege's family, insists that "Lakshman" tutor Bhavani's daughter Urmila (Bindiya Goswami) in music.  Now Ram has to play two roles for Bhavani - the earnest respectful Ram, and the modern loudmouth, Lakshman.  Other improvisations force Ram to enlist an amateur actress, Kamla Shrivastav (Dina Pathak) to introduce to Bhavani as his mother.  Soon, when Kamla runs into Bhavani at a party, she too invents a twin for herself, so as not to blow Ram's cover.  It's not long before all hell breaks loose when Bhavani decides that the hard-working Ram is a perfect match for Urmila - she, of course, prefers the free-spirited Lakshman.

It's hard to say what is most to love about Golmaal.  It's crisply paced and smartly scripted, full of the sort of linguistic humor that I especially love (and that was also in Chupke chupke) - Ram graces Bhavani with such upstanding shuddh Hindi that Bhavani can't quite keep up.  The physical humor is hilarious and masterfully performed.  Utpal Dutt pulls a diverse array of magnificent expressions as Bhavani - wistful as Ram praises the value of the mustache, horrified as Bhavani listens to Urmila recite shamefully modern lines from a play she is acting in.  Amol Palekar, too, is charming, adorable, and very, very funny as he squirms every which way to convince Bhavani that his two sides are two different people.  One cute running physical joke has Ram constantly tugging at his uncomfortably short kurta.  The kurta itself is the result of another of Hrishida's favorite devices, movie in-jokes.  Ram borrows the kurta from his childhood friend, the actor Deven Verma (playing himself), on a movie set - it's short because it's Asrani's kurta, settled on after rejecting Amitabh Bachchan's as too long, and Sanjeev Kumar's as too wide.

Golmaal has too many funny set-pieces and running jokes to call them all out, but a few bear mention.  In one great sequence, Bhavani's widowed sister overhears Urmila rehearsing the melodramatic lines from her play, and concludes that Urmila is pregnant by a man who is now abandoning her.  Much hilarity ensues.  There are very funny running jokes about Ram's mustache.  Bhavani is nearly obsessed with the idea that the mustache makes the man, and Ram is equally attached to his own, which he sacrifices for the sake of his double life.  Every interaction between Ram and Bhavani is hilarious, whether showcasing Bhavani's over-the-top paternal affection for Ram Prasad, or his equally extreme disdain for Lakshman. 

The greatest joy of Golmaal is that even in its comedy of errors, it resonates with everyday experience.  Ram is neither as earnest as Ram Prasad, nor as breezy as Lakshman - his true character is a blend of both.  But most of us find that different aspects of who we are emerge in our interactions with various people.  I do not show the same face to the head of my department at work as I do to my closest friend.  By amplifying this sort of common multiplicity and taking it to an absurd extreme, Hrishikesh Mukherjee makes Ram - and Golmaal - supremely relatable.  Add in a lot of laughs and a few catchy songs, and the sum is just a thoroughly delightful movie.

\end{document}
